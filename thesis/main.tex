\chapter{Stan początkowy projektu}
\label{cha:pocz}

\section{Architektura}
\section{Budowanie}
\section{Dostarczanie i uruchamianie}
\section{Kontrola wersji}


\chapter{Stan docelowy projektu}
\label{cha:docel}


\chapter{Ograniczenia dostępnej infrastruktury}
\label{cha:ogra}

\section{Ograniczone uprawnienia w środowisku docelowym}
\section{Wersje kompilatorów i interpreterów}
\section{Wersje narzędzia budującego CMake}
\section{Związek projektu z wersją jądra systemu}

\chapter{Wykonane prace}
\label{cha:prace}

\section{Wykorzystanie funkcjonalności portalu Gitlab wspierających zarządzanie projektem}
\section{Migracja projektu do systemu kontroli wersji Git i zmiany w architekturze}
\section{Zastosowanie podejścia CI/CD}
\section{Zmiana sposobu budowania aplikacji}
\section{Budowanie i dystrybucja sterownika oraz aplikacji testującej}
\section{Maszyna wirtualna oraz konteneryzacja - Docker}
\section{Pomniejsze prace}
\subsection{Integracja bibliotek napisanych w języku C z aplikacją w C++}
\subsection{Integracja zewnętrznej biblioteki dynamicznej z użyciem narzędzia CMake}
\section{Dokumentacja projektu}

\chapter{Dalsza ścieżka rozwoju projektu}
\label{cha:dalsze}

\section{Wprowadzenie zautomatyzowanego systemu testowania projektu}
\section{Migracja do nowego standardu języka C++}
\section{Automatyzacja procesu publikowania produktu}

