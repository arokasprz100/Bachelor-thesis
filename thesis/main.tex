%%%%%%%%%%%%%%%%%%%%%%%%%%%%%%%%%%%%%%%%%%%%%%%%%%%%%%%%%%%%%%%%%%%%%%%%%%%%%%%%%%%%%
%%%%%%%%%%%%%%%%%%%%%%%%%%%%%%%%%%%%%%%%%%%%%%%%%%%%%%%%%%%%%%%%%%%%%%%%%%%%%%%%%%%%%
%%%%%%%%%%%%%%%%%%%%%%%%%%%%%%%%%%%%%%%%%%%%%%%%%%%%%%%%%%%%%%%%%%%%%%%%%%%%%%%%%%%%%
%%%%%%%%%%%%%%%%%%%%%%%%%%% STAN POCZATKOWY PROJEKTU %%%%%%%%%%%%%%%%%%%%%%%%%%%%%%%%
%%%%%%%%%%%%%%%%%%%%%%%%%%%%%%%%%%%%%%%%%%%%%%%%%%%%%%%%%%%%%%%%%%%%%%%%%%%%%%%%%%%%%
%%%%%%%%%%%%%%%%%%%%%%%%%%%%%%%%%%%%%%%%%%%%%%%%%%%%%%%%%%%%%%%%%%%%%%%%%%%%%%%%%%%%%
%%%%%%%%%%%%%%%%%%%%%%%%%%%%%%%%%%%%%%%%%%%%%%%%%%%%%%%%%%%%%%%%%%%%%%%%%%%%%%%%%%%%%
\chapter{Stan początkowy projektu}
\label{cha:pocz}

\section{Architektura}

\section{Budowanie}

\section{Dostarczanie i uruchamianie}

\section{Kontrola wersji}


\chapter{Stan docelowy projektu}
\label{cha:docel}

%%%%%%%%%%%%%%%%%%%%%%%%%%%%%%%%%%%%%%%%%%%%%%%%%%%%%%%%%%%%%%%%%%%%%%%%%%%%%%%%%%%%%
%%%%%%%%%%%%%%%%%%%%%%%%%%%%%%%%%%%%%%%%%%%%%%%%%%%%%%%%%%%%%%%%%%%%%%%%%%%%%%%%%%%%%
%%%%%%%%%%%%%%%%%%%%%%%%%%%%%%%%%%%%%%%%%%%%%%%%%%%%%%%%%%%%%%%%%%%%%%%%%%%%%%%%%%%%%
%%%%%%%%%%%%%%%%%%%%%%%%%%% OGRANICZENIA INFRASTRUKTURY %%%%%%%%%%%%%%%%%%%%%%%%%%%%%
%%%%%%%%%%%%%%%%%%%%%%%%%%%%%%%%%%%%%%%%%%%%%%%%%%%%%%%%%%%%%%%%%%%%%%%%%%%%%%%%%%%%%
%%%%%%%%%%%%%%%%%%%%%%%%%%%%%%%%%%%%%%%%%%%%%%%%%%%%%%%%%%%%%%%%%%%%%%%%%%%%%%%%%%%%%
%%%%%%%%%%%%%%%%%%%%%%%%%%%%%%%%%%%%%%%%%%%%%%%%%%%%%%%%%%%%%%%%%%%%%%%%%%%%%%%%%%%%%

\chapter{Ograniczenia dostępnej infrastruktury}
\label{cha:ogra}
Z uwagi na silny związek oprogramowania GGSS z infrastrukturą CERN oraz wymóg zapewnienia możliwości budowania projektu na należących do niej maszynach, przed autorami postawiony został szereg ograniczeń związanych z możliwymi do użycia technologiami oraz sposobem wykonywania pewnych operacji. Niniejszy rozdział stanowi opis najważniejszych z tych ograniczeń z uwzględnieniem ich wpływu na obraną przez autorów pracy ścieżkę rozwoju projektu.


\section{Ograniczone uprawnienia w środowisku docelowym}


\section{Wersje kompilatorów i interpreterów}
Dostępne wersje kompilatorów i interpreterów stanowią jeden z kluczowych czynników, który należy uwzględnić podczas wprowadzania zmian w istniejącym systemie, ponieważ definiują one możliwy do wykorzystania podzbiór technologii. W kontekście systemu GGSS ograniczenia te dotyczą przede wszystkim kompilatora języka C++ oraz interpretera języka Python. 

\paragraph*{Wersja kompilatora języka C++}\mbox{}\\
Dostępna w ramach infrastruktury projektu wersja kompilatora języka C++ to \textbf{g++ (GCC) 4.8.5}. Wspiera ona w pełni standard C++11, czyli funkcjonalności takie, jak referencje do r-wartości, wyrażenia lambda czy zakresowa pętla for. \textbf{STRONA Z NOWOSCIAMI W WERSJACH} Wersja ta nie wspiera niestety nowszych wydań języka (C++14/17).

\paragraph*{Wersja interpretera języka Python}\mbox{}\\
Domyślną wersją Pythona jest \textbf{Python 2.7.5}, jednak dostępny jest również Python 3 (w wersji \textbf{Python 3.6.8}). Z uwagi na wspomniany wcześniej koniec oficjalnego wsparcia dla Pythona 2, który ma nadejść wraz z początkiem 2020 roku, naturalnym jest więc wybór wersji 3. Infrastruktura projektu posiada jednak znaczące braki jeśli chodzi o dostępne dla wersji 3 biblioteki zewnętrzne - domyślnie nie jest np. dostępna biblioteka \textit{Beautiful Soup}, slużąca do przetwarzania dokumentów w formacie HTML. Niektóre popularne bibliteki i frameworki (np. \textit{PyTest} - wykorzystywany do przeprowadzania testów oprogramowania) nie są dostępne dla obu wersji Pythona. Taka sytuacja wymusza więc wykorzystanie narzędzia \textit{virtualenv} w celu ich instalacji w odizolowanym środowisku, nie mającym wpływu na infrastrukturę CERN-u.

\section{Wersja narzędzia budującego CMake}
Dostępna wersja narzędzia CMake stanowiła zdaniem autorów największe ograniczenie w czasie prac nad projektem. Na maszynach docelowych dostępna jest jedynie stara wersja \textbf{2.8.12.2}. Nowsza wersja (\textbf{3.14.6}) dostępna jest na niektórych z komputerów, jednak z uwagi na konieczność zachowania kompatybilności ze wspomnianymi maszynami docelowymi, nie było możliwe jej użycie. Stosowanie wersji o numerze niższym niż \textbf{3.0} skutkuje szeregiem ograniczeń - brakuje w niej wielu funkcjonalności pozwalających na stosowanie ogólnoprzyjętych dziś praktyk, jak np. określenie zakresu wersji narzedzia CMake, w którym powinna mieścić się używana wersja, by projekt można było bez problemu zbudować, czy wsparcie dla instrukcji \textit{target\_link\_directories}. \textbf{https://cliutils.gitlab.io/modern-cmake/chapters/intro/newcmake.html}

\section{Związek projektu z wersją jądra systemu}


%%%%%%%%%%%%%%%%%%%%%%%%%%%%%%%%%%%%%%%%%%%%%%%%%%%%%%%%%%%%%%%%%%%%%%%%%%%%%%%%%%%%%
%%%%%%%%%%%%%%%%%%%%%%%%%%%%%%%%%%%%%%%%%%%%%%%%%%%%%%%%%%%%%%%%%%%%%%%%%%%%%%%%%%%%%
%%%%%%%%%%%%%%%%%%%%%%%%%%%%%%%%%%%%%%%%%%%%%%%%%%%%%%%%%%%%%%%%%%%%%%%%%%%%%%%%%%%%%
%%%%%%%%%%%%%%%%%%%%%%%%%%%%%%%%%%%%%%%%%%%%%%%%%%%%%%%%%%%%%%%%%%%%%%%%%%%%%%%%%%%%%
%%%%%%%%%%%%%%%%%%%%%%%%%%%%%%%%%%%%%%%%%%%%%%%%%%%%%%%%%%%%%%%%%%%%%%%%%%%%%%%%%%%%%
%%%%%%%%%%%%%%%%%%%%%%%%%%%%%%%%%%%%%%%%%%%%%%%%%%%%%%%%%%%%%%%%%%%%%%%%%%%%%%%%%%%%%

\chapter{Wykonane prace}
\label{cha:prace}

\section{Wykorzystanie funkcjonalności portalu Gitlab wspierających zarządzanie projektem}
\section{Migracja projektu do systemu kontroli wersji Git i zmiany w architekturze}
\section{Zastosowanie podejścia CI/CD}
\section{Zmiana sposobu budowania aplikacji}
\section{Budowanie i dystrybucja sterownika oraz aplikacji testującej}
\section{Maszyna wirtualna oraz konteneryzacja - Docker}
\section{Pomniejsze prace}
\subsection{Integracja bibliotek napisanych w języku C z aplikacją w C++}
\subsection{Integracja zewnętrznej biblioteki dynamicznej z użyciem narzędzia CMake}
\section{Dokumentacja projektu}

\chapter{Dalsza ścieżka rozwoju projektu}
\label{cha:dalsze}

\section{Wprowadzenie zautomatyzowanego systemu testowania projektu}
\section{Migracja do nowego standardu języka C++}
\section{Automatyzacja procesu publikowania produktu}

