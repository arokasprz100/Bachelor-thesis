\chapter{Zastosowane technologie}
\label{cha:teoria}

\section{Język C++}
\textit{C++} jest kompilowanym językiem programowania ogólnego przeznaczenia \textbf{BJARNE} opartym o statyczne typowanie. Został stworzony jako obiektowe rozszerzenie języka C (z którym jest w dużej mierze wstecznie kompatybilny), lecz wraz z rozwojem pojawiło się w nim wsparcie dla innych paradygmatów, w tym generycznego i funkcyjnego. Sprawiło to, że język ten stał się bardzo wszechstronny - pozwala zarówno na szybkie wykonywanie operacji niskopoziomowych \textbf{BJARNE strona 41}, jak i na tworzenie wysokopoziomowych abstrakcji\textbf{BJARNE, PRACA GOSCIA}. Dodatkową cechą wyróżniającą C++ wśród innych języków umożliwiających programowanie obiektowe jest jego wysoka wydajność.

\paragraph*{Standardy języka} \mbox{} \
Od ostatnich kilku lat C++ przechodzi proces intensywnego rozwoju - od 2011 roku pojawiły się trzy nowe standardy tego języka, a kolejny przewidziany jest na rok 2020. Wspomniane nowe standardy to:
\begin{itemize}
\item C++11 - wprowadza funkcjonalności takie jak: wsparcie dla wielowątkowości, wyrażenia lambda, referencje do \textit{r-wartości}, biblioteka do obsługi wyrażeń regularnych, dedukcja typów za pomocą słowa kluczowego \textit{auto} czy pętla zakresowa. Standard ten uważany jest za przełom w rozwoju języka.
\item C++14 - rozszerza zmiany wprowadzone w C++11. Nie zawiera tak wielu przełomowych zmian jak poprzedni standard - twórcy skupili się na poprawie istniejących błędów oraz rozwoju istniejących rozwiązań \textbf{wiki:} np. dedukcji typu zwracanego z funkcji za pomocą słowa kluczowego \textit{auto}.
\item C++17 - wprowadza m.in. nowe typy danych (\textit{std::variant} czy \textit{std::optional}, algorytmy współbieżne, 
\end{itemize}
\section{Biblioteki statyczne i dynamiczne}
\section{Narzędzie CMake}
\section{Język Python}
\section{Powłoka systemu operacyjnego - Bash}
Powłoka systemu jest programem, którego głównym zadaniem jest udostępnienie interfejsu umożliwiającego łatwy dostęp do funkcji systemu operacyjnego. Nazwę \textit{powłoka} zawdzięcza temu, że jest warstwą okalającą system operacyjny. Najczęściej spotykanym rodzajem powłoki są tzw. interfejsy z wierszem poleceń (ang. command-line interface). Polecenia wprowadzane są do nich w modzie interaktywnym, tj. wykonywane są one w momencie wprowadzenia końca linii.

\begin{lstlisting}[language=bash,caption={Komenda wypisująca tekst na standardowe wyjście wykonana z linii poleceń}]
user@host:~$ echo "interfejs z linią poleceń"
interfejs z linią poleceń
user@host:~$
\end{lstlisting}

Bash, czyli \textbf{Bourne Again Shell} jest powłoką systemu początkowo napisaną dla systemu operacyjnego GNU. Obecnie Bash jest kompatybilny z większością systemów Unixowych, gdzie zwykle jest powłoką domyślną oraz posiada kilka portów na inne platformy, tj.: MS-DOS, OS/2, Windows. WSTAWIC REFERENCJE DO www.gnu.org\/software\/bash\/manual\/bash.html\#What-is-Bash\_003f Oprócz pełnienia wyżej wymienionej funkcji, Bash jest również językiem programowania pozwalającym na tworzenie skryptów, które są kolejną metodą wprowadzania poleceń do powłoki systemu.\par
Korzystając z języka skryptowego powłoki Bash jesteśmy w stanie zawrzeć dodatkową logikę podczas wykonywania komend. Wspiera on takie struktury jak: instrukcje warunkowe, pętle, operacje logiczne oraz arytmetyczne. Aby wykorzystać Bash w skrypcie należy na początku pliku zamieścić zapis \textbf{\#!/bin/bash}, gdzie \textbf{/bin/bash} to ścieżka do pliku interpretera Bash. Zachowanie skryptu jesteśmy w stanie uzależnić od argumentów wykonania. Ich obsługa odbywa się za pomocą zapisu \textbf{\$?}, gdzie \textbf{?} jest to numer porządkowy argumentu liczony od 0.

\begin{lstlisting}[label={lst:prostySkrypt},language=bash,caption={Skrypt wykorzystujący argumenty wejściowe, instrukcję warunkową oraz polecenie echo}]
#!/bin/bash
if [ $1 == "argumenty" ]; then
        echo "Argument 0.: $0"
        echo "Argument 1.: $1"
else
        echo "Nieznane polecenie"
fi
\end{lstlisting}

\begin{lstlisting}[language=bash,caption={Przykład działania Skryptu z Listingu \ref{lst:prostySkrypt}}]
user@host:~$ /home/user/prostySkrypt.sh argumenty
Argument 0.: /home/user/prostySkrypt.sh
Argument 1.: argumenty
\end{lstlisting}

Bash posiada wiele poleceń, które pozwalają na wykonywanie zarówno podstawowych, jak i bardziej zaawansowanych czynności, np.: obsługa plików, obsługa systemu katalogów, zarządzanie kontami, uprawnieniami, itd.\par
Bash posiada również wiele zaawansowanych funkcjonalności, które pozwalają na kontrolowanie przepływu informacji w trakcie wykonywania poleceń. Przykładem jest wpisywanie tekstu do pliku ukazane na Listingu \ref{lis:zapisDoPliku}.

\begin{lstlisting}[label={lis:zapisDoPliku},language=bash,caption={Przykład zapisu tekstu do pliku}]
user@host:~$ echo "Ten napis zostanie zapisany do pliku plik.txt" > plik.txt
user@host:~$ cat plik.txt
Ten napis zostanie zapisany do pliku plik.txt
\end{lstlisting}

W celu zapisania tekstu do pliku należy na standardowe wyjście przekazać napis za pomocą komendy \textbf{echo}, a następnie przekierować za pomocą zapisu \textbf{>}, który poprzedza nazwę pliku docelowego. W wyniku działania zawartośc pliku \textbf{plik.txt} zostanie nadpisana, a w przypadku gdy takiego pliku nie ma, to zostanie on utworzony i uzupełniony o napis.\par


\section{System kontroli wersji Git i portal Gitlab}
System kontroli wersji Git jest oprogramowaniem służącym do śledzenia i zarządzania zmianami w plikach projektowych. W przypadku Git'a, aby zarejestrować pliki projektowe w celu ich śledzenia należy wykonać kilka czynności. Po pierwsze wymagane jest utworzenie repozytorium. Sprowadza się ono do wykonania odpowiedniej komendy Git'a wewnątrz folderu projektu, tj. \textbf{git init}. Podczas działania komendy wewnątrz folderu, w którym wywołaliśmy ww. polecenie, inicjowany jest ukryty folder \textbf{.git}
. Jest on odpowiedzialny za przechowywanie konfiguracji dla tego repozytorium oraz zapisywanie informacji o wszystkich zmianach dokonanych w projekcie.

\begin{lstlisting}[language=bash,caption={Inicjalizacja repozytorium git}]
user@host:/ścieżka/do/projektu$ git init
Initialized empty Git repository in /ściezka/do/projektu
user@host:/ściezka/do/projektu$ ls .git
branches  config  description  HEAD  hooks  info  objects  refs
\end{lstlisting}

Taka inicjalizacja nie spowoduje żadnego dodatkowego działania oprócz utworzenia repozytorium. Żadne pliki nie są jeszcze poddawane rewizji. W celu rejestracji plików należy wykonać jeszcze kilka kroków. Pierwszym z nich jest wykonanie komendy \textbf{git add}, która poprzedza nazwę plików lub folderów, które chcemy poddać wersjonowaniu. Elementy te zostają dodane do tzw. poczekalni, czyli są one kandydatami do utworzenia kolejnej rewizji. Przydatną komendą w tym przypadku jest również \textbf{git status} pozwalająca na sprawdzenie obecnego stanu repozytorium. Wyświetla ono krótkie podsumowanie nt. nowych plików, usuniętych plików oraz plików zmodyfikowanych. Informuje nas również o tym, które pliki są brane pod uwagę do utworzenia kolejnej rewizji.

\begin{lstlisting}[language=bash,caption={Dodawanie elementów do poczekalni}]
user@host:/ściezka/do/projektu# git add plik1 folder1
user@host:/ściezka/do/projektu# git status
On branch master

No commits yet

Changes to be committed:
  (use "git rm --cached <file>..." to unstage)

        new file:   folder1/plik3
        new file:   folder1/plik4
        new file:   plik1

Untracked files:
  (use "git add <file>..." to include in what will be committed)

        plik2
\end{lstlisting}

Tworzenie nowej wersji w ramach repozytorium odbywa się za pomocą komendy \textbf{git commit}. Sprowadza się do 'zamrożenia' obecnych wersji plików zarejestrowanych do rewizji oraz przypisanie im wspólnego, unikalnego dla każdej z nich, identyfikatora. Git udostępnia komendy pozwalające na przeglądanie oraz przywracanie plików do wcześniej utworzonych wersji. Listing \ref{lis:pierwszaRewizja} przedstawia utworzenie nowej wersji oraz wyświetlenie podsumowania o utworzonych do tej pory rewizjach.

\begin{lstlisting}[label={lis:pierwszaRewizja}, language=bash,caption={Utworzenie nowej rewizji}]
user@host:/sciezka/do/projektu# git commit -m "Pierwsza rewizja"
user@host:/sciezka/do/projektu# git log 
commit 1d2445e961beb25940dffa9d73963f887ee553ad
Author: user <user@host.localdomain>
Date:   Wed Dec 4 17:29:55 2019 +0100

    Pierwsza rewizja
\end{lstlisting}

Przejścia między rewizjami nie powodują utraty danych, gdyż zachowywana jest informacja o stanie plików dla każdej z nich, co ukazuje Listing \ref{lis:drugaRewizja}. Tworzona jest nowa rewizja zawierająca dodatkowo \textbf{plik2}, natomiast po powrocie do poprzedniej wersji plik ten nie występuje. Gdy powrócimy do nowszej wersji ponownie pojawi się \textbf{plik2}.

\begin{lstlisting}[label={lis:drugaRewizja}, language=bash,caption={Podsumowanie rewizji, powrót do starszej wersji}]
user@host:/sciezka/do/projektu# git add plik2
user@host:/sciezka/do/projektu# git commit -m "Druga rewizja"
user@host:/sciezka/do/projektu# git log 
commit b58836df55fc2a8eb2a43aa96273853776924807
Author: user <user@host.localdomain>
Date:   Wed Dec 4 17:30:10 2019 +0100

    Druga rewizja

commit 1d2445e961beb25940dffa9d73963f887ee553ad
Author: user <user@host.localdomain>
Date:   Wed Dec 4 17:29:55 2019 +0100

    Pierwsza rewizja

user@host:/sciezka/do/projektu# ls
folder1  plik1  plik2
user@host:/sciezka/do/projektu# git checkout 1d2445e961beb25940dffa9d73963f887ee553ad
user@host:/sciezka/do/projektu# ls
folder1  plik1
user@host:/sciezka/do/projektu# git checkout b58836df55fc2a8eb2a43aa96273853776924807
user@host:/sciezka/do/projektu# ls
folder1  plik1  plik2
\end{lstlisting}

\section{Manager pakietów - RPM}
\section{Technologie wirtualizacji i konteneryzacji}
