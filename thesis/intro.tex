\chapter{Wstęp}
\label{cha:wstep}

\section{Wprowadzenie do systemu GGSS}
Detektor ATLAS (\textit{A Toroidal LHC ApparatuS}), znajdujący się w~Europejskim Ośrodku Badań Jądrowych CERN, jest jednym z~detektorów pracujących przy Wielkim Zderzaczu Hadronów (\textit{LHC - Large Hadron Collider}). Pełni on kluczową rolę w~rozwoju fizyki cząstek elementarnych, w~szczególności badania przy nim prowadzone doprowadziły do potwierdzenia istnienia tzw. \textit{bozonu Higgsa} w~roku 2012 \cite{AtlasWikipedia}. \par

Detektor ATLAS charakteryzuje się budową warstwową - składa się z~kilku pod-detektorów \cite{AtlasAGH}. Jednym z~nich jest Detektor Wewnętrzny (\textit{Inner Detector}) składający się z~trzech głównych elementów zbudowanych za pomocą różnych technologii. Elementy te, w~kolejności od położonego najbliżej punktu zderzeń cząstek, to: detektor pikselowy (\textit{Pixel Detector}), krzemowy detektor śladów (\textit{SCT - Semiconductor Tracker}) oraz detektor promieniowania przejścia (\textit{TRT - Transition Radiation Tracker}). Dokładny opis zasad działania całego detektora oraz poszczególnych jego komponentów wykracza poza zakres niniejszego manuskryptu.\par

W kontekście niniejszej pracy kluczowym jest System Stabilizacji Wzmocnienia Gazowego (\gls*{ggss}) dla detektora \textit{TRT}. Jego oprogramowanie jest zintegrowane \cite{AtlasAGH} z~systemem kontroli detektora ATLAS (\textit{DCS - Detector Control System}). W~skład systemu \gls*{ggss} wchodzą zarówno urządzenia takie jak multiplekser i~zasilacz wysokiego napięcia, jak i~rozbudowana warstwa oprogramowania. Autorzy pracy zaprezentują opis zmian, jakie do tej pory wprowadzili w~projekcie \gls*{ggss}. Zmiany te obejmują m.in. sposób budowania aplikacji wchodzących w~skład systemu, ale również automatyzacja prac związanych z~jego utrzymaniem i~użytkowaniem.



\section{Cel pracy}
Przed autorami postawiony został szereg celów do zrealizowania, związanych zarówno ze zdobyciem wymaganej wiedzy domenowej, jak i~przeprowadzeniem modyfikacji oprogramowania systemu \gls*{ggss}. \par

Jednym z~nich było zapoznanie się z~infrastrukturą informatyczną CERN-u. Praca z~oprogramowaniem oparta jest tam o~unikalny ekosystem, mający zapewnić bezpieczeństwo i~stabilność całej infrastruktury, co wiąże się z~wieloma ograniczeniami dotyczącymi m.in. dostępu do komputerów produkcyjnych. Konieczne było więc uzyskanie odpowiednich uprawnień i~zdobycie doświadczenia w~pracy z~tą infrastrukturą. Ze względu na domenę działania systemu \gls*{ggss} celem było również zdobycie wiedzy na temat sposobu pracy przy dużych eksperymentach, na przykładzie eksperymentu ATLAS. Ponadto uczestnictwo w~rozwoju projektu tego typu miało na celu nabycie przez autorów doświadczenia w~pracy w~międzynarodowym środowisku, jakim jest CERN. Kluczowym dla poprawnego przeprowadzenia prac było również zapoznanie się autorów z~zastosowaniem i~podstawami sposobu działania systemu \gls*{ggss}. \par

Oprócz wyżej wymienionych czynności związanych ze zdobyciem podstawowej wiedzy domenowej, celem niniejszej pracy było przeprowadzenie modyfikacji w~warstwie oprogramowania projektu \gls*{ggss}. Do postawionych przed autorami zadań należało zaplanowanie prac i~utworzenie wygodnego, nowoczesnego środowiska do zarządzania projektem informatycznym oraz prostego w~rozwoju, intuicyjnego systemu budowania oprogramowania opartego o~narzędzie \gls*{cmake} \cite{CMakeMain}. Miało to na celu umożliwienie modularyzacji projektu tak, by każdy z~komponentów mógł być niezależnie budowany. Ponadto zadaniem autorów była migracja projektu do systemu kontroli wersji \gls*{git} \cite{GitMain}, stanowiącego ogólnie przyjęty standard we współczesnych projektach informatycznych. W~celu uproszczenia procedury wdrażania projektu w~środowisku produkcyjnym celem autorów było również zautomatyzowanie procesu budowania i~dystrybucji projektu. Na koniec, by umożliwić innym uczestnikom projektu sprawne korzystanie z~nowych rozwiązań, przygotowana miała zostać dokumentacja w~formie krótkich instrukcji oraz zestawów komend. Dokumentacja, z~uwagi na międzynarodowy charakter środowiska w~CERN, miała zostać napisana w~języku angielskim. \par

Niniejszy manuskrypt opisuje przede wszystkim prace związane z~rozwojem oprogramowania przeprowadzone przez autorów. Praca opisuje stan początkowy projektu, założenia dotyczące stanu docelowego oraz wybrane, zdaniem autorów najważniejsze, zadania zrealizowane w~ramach pracy z~oprogramowaniem systemu \gls*{ggss}.
